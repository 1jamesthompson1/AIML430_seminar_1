\documentclass{beamer}
\usepackage[utf8]{inputenc}
\usepackage{graphicx}
\usepackage[backend=biber, style=verbose]{biblatex}
\addbibresource{references.bib}


\title{Presentation of Potential of artificial intelligence in reducing energy and carbon emissions of commercial buildings at scale}
\author{Dwayne Mark (Dwayne) Acosta \\ Mohamed Amine Benaziza \\ David Franz \\ Ray Marange \\ James Thompson}
\date{\today}

\begin{document}

\frame{\titlepage}

\begin{frame}{Introduction}
\framesubtitle{Presented by: Ray Marange}
Climate change is accelerating, and buildings are a major contributor, responsible for 39\% of U.S. primary energy use. With urbanization surging and building stock/demand expected to double by 2060, improving building efficiency is no longer optional but urgent.
While AI has transformed industries such as healthcare and finance, its potential in building energy efficiency remains underexplored. AI demonstrates significant potential to reduce costs, enhance benefits, and improve safety across the building lifecycle. The study investigates how AI can reduce energy consumption and carbon emissions in medium-sized office buildings, offering a scalable framework that could be applied globally.
We will explore four key areas: \textbf{Results}, \textbf{Discussion}, \textbf{Methods}, and \textbf{Takeaways \& Reflections}. The study focuses on medium-sized offices, and the results can be extrapolated to offices of any size.
\end{frame}

\begin{frame}{Results part 1}
\framesubtitle{Presented by: Dwayne Mark Acosta}

\end{frame}

\begin{frame}{Results part 2}
\framesubtitle{Presented by: David Franz}


\end{frame}

\begin{frame}{Discussions}
\framesubtitle{Presented by: Mohamed Amine Benaziza}

\end{frame}

\begin{frame}{Methods}
\framesubtitle{Presented by: James Thompson}

\end{frame}
\end{document}

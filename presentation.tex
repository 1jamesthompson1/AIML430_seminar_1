\documentclass{beamer}
\usepackage[utf8]{inputenc}
\usepackage{graphicx}
\usepackage[backend=biber, style=verbose]{biblatex}
\addbibresource{references.bib}


\title{Presentation of Potential of artificial intelligence in reducing energy and carbon emissions of commercial buildings at scale}
\author{Dwayne Mark (Dwayne) Acosta \\ Mohamed Amine Benaziza \\ David Franz \\ Ray Marange \\ James Thompson}
\date{\today}

\begin{document}

\frame{\titlepage}

\begin{frame}{Introduction}
\framesubtitle{Presented by: Ray Marange}
Climate change is accelerating, and buildings are a major contributor, responsible for 39\% of U.S. primary energy use. With urbanization surging and building stock/demand expected to double by 2060, improving building efficiency is no longer optional but urgent.
While AI has transformed industries such as healthcare and finance, its potential in building energy efficiency remains underexplored. AI demonstrates significant potential to reduce costs, enhance benefits, and improve safety across the building lifecycle. The study investigates how AI can reduce energy consumption and carbon emissions in medium-sized office buildings, offering a scalable framework that could be applied globally.
We will explore four key areas: \textbf{Results}, \textbf{Discussion}, \textbf{Methods}, and \textbf{Takeaways \& Reflections}. The study focuses on medium-sized offices, and the results can be extrapolated to offices of any size.
\end{frame}

\begin{frame}{Results part 1}
\framesubtitle{Presented by: Dwayne Mark Acosta}

\end{frame}

\begin{frame}{Results part 2}
\framesubtitle{Presented by: David Franz}


\end{frame}

\begin{frame}{Discussions}
\framesubtitle{Presented by: Mohamed Amine Benaziza}

% Just demonstrating what it would looke like for Anime to get a feel for latex syntax.
Method and Scope
\begin{itemize}
    \item Uses engineering + energy-simulation rather than one specific AI technology to estimate how AI can boost building-energy efficiency and cut carbon.
    \item The paper focuses on a medium-office as an example, yet methodology is transferable to other commercial buildings with adjustments.
\end{itemize}
\pause % This means that it will render this slide with the bits above first and then the bits below will appear after going to the next slide. You can add as many as you want. but for lists you can use a special thing below.
Why AI helps
\begin{itemize}[<+->] % Checkout the pdf presentation the list appears incrementally.
    \item Data-driven modeling can tailor solutions and lower costs, accelerating adoption of high-efficiency / net-zero-energy buildings (HEEBs \& NZEBs).
    \item Advanced control models (deep learning, reinforcement learning) could refine accuracy in future work.
\end{itemize}

\end{frame}

\begin{frame}{Methods}
\framesubtitle{Presented by: James Thompson}

\end{frame}
\end{document}

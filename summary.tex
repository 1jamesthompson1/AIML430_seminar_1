\documentclass[conference,a4paper]{IEEEtran}
\IEEEoverridecommandlockouts
% The preceding line is only needed to identify funding in the first footnote. If that is unneeded, please comment it out.
%Template version as of 6/27/2024

\usepackage{cite}
\usepackage{amsmath,amssymb,amsfonts}
\usepackage{algorithmic}
\usepackage{graphicx}

\usepackage{textcomp}
\usepackage{tikz}
\usepackage{booktabs}
\usepackage{xcolor}
\usepackage{hyperref}
\def\BibTeX{{\rm B\kern-.05em{\sc i\kern-.025em b}\kern-.08em
    T\kern-.1667em\lower.7ex\hbox{E}\kern-.125emX}}
\begin{document}

% Custom figure command: pass filename and caption
\newcommand{\cfigure}[2]{%
  \begin{figure}[h]
    \centering
    \includegraphics[width=\linewidth]{figures/#1.png}%
    \caption{#2}%
    \label{fig:#1}%
  \end{figure}%

}
\title{Summary of Potential of artificial intelligence in reducing energy and carbon emissions of commercial buildings at scale}

\author{Dwayne Mark (Dwayne) Acosta \\ Mohamed Amine (Mohamed) Benaziza \\ David Franz \\ Ray Marange \\ James Thompson\\
\textit{Victoria University of Wellington}\\}
\date{\today}

\maketitle

\section*{Introduction}
\textit{Summary by Ray Marange}
Ding et Al\footcite{dingPotentialArtificialIntelligence2024} explores the potential for AI \dots.

Climate change is accelerating, and buildings are a major contributor, responsible for 39\% of U.S. primary energy use. With urbanization surging and building stock/demand expected to double by 2060, improving building efficiency is no longer optional but urgent.
While AI has transformed industries such as healthcare and finance, its potential in building energy efficiency remains underexplored. AI demonstrates significant potential to reduce costs, enhance benefits, and improve safety across the building lifecycle. This study investigates how AI can reduce energy consumption and carbon emissions in medium-sized office buildings, offering a scalable framework that could be applied globally.
We will explore four key areas: \textbf{Results}, \textbf{Discussion}, \textbf{Methods}, and \textbf{Takeaways \& Reflections}. The study focuses on medium-sized offices, and the results can be extrapolated to offices of any size.
\end{frame}

\begin{frame}{Results part 1}
\framesubtitle{Presented by: Dwayne Mark Acosta}

\end{frame}
This paper \cite{dingPotentialArtificialIntelligence2024} \dots

\section*{AI's impact on energy and emission reductions}
\textit{Summary by Dwayne Mark Acosta}

Talking about a cool figure \ref{fig:cool-figure}. What a cool figure it is!

\cfigure{cool-figure}{Different energy use scenarios.}

\section*{AI's reduces emissions of buildings}
\textit{Summary by David Franz}

Need to add a table as its better than lots of words. Table \ref{tab:energy-emissions} shows the energy use and CO$_2$ emissions for different scenarios.

\begin{table}
\centering
\begin{tabular}{|c|c|c|}
\hline
\textbf{Scenario} & \textbf{Energy Use (kWh/m$^2$)} & \textbf{CO$_2$ Emissions (kg/m$^2$)} \\
\hline 
Baseline & 200 & 50 \\
\hline
AI Optimized & 150 & 30 \\
\hline
\end{tabular}
\label{tab:energy-emissions}
\end{table}

\section*{Discussion}
\textit{Summary by Mohamed Amine Benaziza}

\section*{Methods}
\textit{Summary by James Thompson}

\bibliographystyle{IEEEtran}
\bibliography{references}

\appendix



\end{document}

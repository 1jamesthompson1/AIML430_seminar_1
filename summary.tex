\documentclass[conference,a4paper]{IEEEtran}
\IEEEoverridecommandlockouts
% The preceding line is only needed to identify funding in the first footnote. If that is unneeded, please comment it out.
%Template version as of 6/27/2024

\usepackage{cite}
\usepackage{amsmath,amssymb,amsfonts}
\usepackage{algorithmic}
\usepackage{graphicx}

\usepackage{textcomp}
\usepackage{tikz}
\usepackage{booktabs}
\usepackage{xcolor}
\usepackage{hyperref}
\def\BibTeX{{\rm B\kern-.05em{\sc i\kern-.025em b}\kern-.08em
    T\kern-.1667em\lower.7ex\hbox{E}\kern-.125emX}}
\begin{document}

% Custom figure command: pass filename and caption
\newcommand{\cfigure}[2]{%
  \begin{figure}[h]
    \centering
    \includegraphics[width=\linewidth]{figures/#1.png}%
    \caption{#2}%
    \label{fig:#1}%
  \end{figure}%

}
\title{Summary of Potential of artificial intelligence in reducing energy and carbon emissions of commercial buildings at scale}

\author{Dwayne Mark (Dwayne) Acosta \\ Mohamed Amine (Mohamed) Benaziza \\ David Franz \\ Ray Marange \\ James Thompson\\
\textit{Victoria University of Wellington}\\}
\date{\today}

\maketitle

\section*{Introduction}
\textit{Summary by Ray Marange}
Ding et Al\footcite{dingPotentialArtificialIntelligence2024} explores the potential for AI \dots.

Climate change is accelerating, and buildings are a major contributor, responsible for 39\% of U.S. primary energy use. With urbanization surging and building stock/demand expected to double by 2060, improving building efficiency is no longer optional but urgent.
While AI has transformed industries such as healthcare and finance, its potential in building energy efficiency remains underexplored. AI demonstrates significant potential to reduce costs, enhance benefits, and improve safety across the building lifecycle. This study investigates how AI can reduce energy consumption and carbon emissions in medium-sized office buildings, offering a scalable framework that could be applied globally.
We will explore four key areas: \textbf{Results}, \textbf{Discussion}, \textbf{Methods}, and \textbf{Takeaways \& Reflections}. The study focuses on medium-sized offices, and the results can be extrapolated to offices of any size.
\end{frame}

\begin{frame}{Results part 1}
\framesubtitle{Presented by: Dwayne Mark Acosta}

\end{frame}
This paper \cite{dingPotentialArtificialIntelligence2024} \dots

\section*{AI's impact on energy and emission reductions}
\textit{Summary by Dwayne Mark Acosta}

% Talking about a cool figure \ref{fig:cool-figure}. What a cool figure it is!
% \cfigure{cool-figure}{Different energy use scenarios.}
% David- I added this figure to the next section since this corresponds to the paper

\section*{AI's reduces emissions of buildings}
\textit{Summary by David Franz}

\subsection*{Primary focus of modeling}

The paper focuses on two ways that AI can reduce the emissions of buildings.

\begin{enumerate}
    \item By helping scale up the technologies and speed adoption by reducing the construction and labor costs;
    \item By helping reduce emissions in ongoing maintenance and any new construction over the entire building's lifetime.
\end{enumerate}

\\

\subsection*{Scenarios simulated}
The paper uses the results gained from the previous section to \textbf{simulate six scenarios}. The data is used to estimate parameters for use with complex simulation software to attempt to model the potential lifetime impact on emissions.
\begin{enumerate}
    \item Frozen with current building efficiency;
    \item BAU without AI;
    \item BAU with AI;
    \item Three policy-driven scenarios promoting high-efficiency energy buildings and net-zero energy buildings, and other policy implementation to achieve zero emissions by 2050.
\end{enumerate}

\\

\subsection*{Simulation results}
The results of the simulation are shown below.

Table \ref{tab:energy-emissions} shows the energy use and CO$_2$ emissions for different scenarios.

\begin{table}
\centering
\begin{tabular}{|c|c|c|}
\hline
\textbf{Scenario} & \textbf{Energy Use (kWh/m$^2$)} & \textbf{CO$_2$ Emissions (kg/m$^2$)} \\
\hline 
Baseline & 200 & 50 \\
\hline
AI Optimized & 150 & 30 \\
\hline
\end{tabular}
\label{tab:energy-emissions}
\end{table}

\cfigure{cool-figure}{Different energy use scenarios.}

\\

\subsection*{Key insight}

\textit{“The scenario with AI leads to a higher market share of HEEBs and NZEBs over time compared with the scenario without AI. This trend continues until the market share of net NZEBs reaches its maximum share.”}

The paper asserts that using AI in the ways that they propose always leads to a higher market share of efficient buildings. 

\\

\subsection*{Thoughts on result}
The paper examines various scenarios with some amount of estimation for unknowns, so each individual simulation is unlikely to be exactly right. However, the fact that all scenarios trend in a downward direction for energy use and $CO_2$ emissions suggest it is highly likely that AI would have some impact on building emissions, but the current lack of data leading to necessary estimation means that the current degree of this impact is still unclear.

\\

\section*{Discussion}
\textit{Summary by Mohamed Amine Benaziza}

\section*{Methods}
\textit{Summary by James Thompson}

\bibliographystyle{IEEEtran}
\bibliography{references}

\appendix



\end{document}

\documentclass[conference,a4paper]{IEEEtran}
\IEEEoverridecommandlockouts
% The preceding line is only needed to identify funding in the first footnote. If that is unneeded, please comment it out.
%Template version as of 6/27/2024

\usepackage{cite}
\usepackage{amsmath,amssymb,amsfonts}
\usepackage{algorithmic}
\usepackage{graphicx}

\usepackage{textcomp}
\usepackage{tikz}
\usepackage{booktabs}
\usepackage{xcolor}
\usepackage{hyperref}
\def\BibTeX{{\rm B\kern-.05em{\sc i\kern-.025em b}\kern-.08em
    T\kern-.1667em\lower.7ex\hbox{E}\kern-.125emX}}
\begin{document}

% Custom figure command: pass filename and caption
\newcommand{\cfigure}[2]{%
  \begin{figure}[h]
    \centering
    \includegraphics[width=\linewidth]{figures/#1.png}%
    \caption{#2}%
    \label{fig:#1}%
  \end{figure}%

}
\title{Summary of Potential of artificial intelligence in reducing energy and carbon emissions of commercial buildings at scale}

\author{Dwayne Mark (Dwayne) Acosta \\ Mohamed Amine (Mohamed) Benaziza \\ David Franz \\ Ray Marange \\ James Thompson\\
\textit{Victoria University of Wellington}\\}
\date{\today}

\maketitle

\section*{Introduction}
\textit{Summary by Ray Marange}
Climate change is accelerating, and buildings are a major contributor, responsible for 39\% of U.S. primary energy use. With urbanization surging and building stock/demand expected to double by 2060, improving building efficiency is no longer optional but urgent.
While AI has transformed industries such as healthcare and finance, its potential in building energy efficiency remains underexplored. AI demonstrates significant potential to reduce costs, enhance benefits, and improve safety across the building lifecycle. This study \cite{dingPotentialArtificialIntelligence2024} investigates how AI can reduce energy consumption and carbon emissions in medium-sized office buildings, offering a scalable framework that could be applied globally.
We will explore four key areas: \textbf{Results}, \textbf{Discussion}, \textbf{Methods}, and \textbf{Takeaways \& Reflections}. The study focuses on medium-sized offices, and the results can be extrapolated to offices of any size.

\section*{AI's impact on energy and emission reductions}
\textit{Summary by Dwayne Mark Acosta}

\section*{AI's reduces emissions of buildings}
\textit{Summary by David Franz}

\subsection*{Primary focus of modeling}

The paper focuses on two ways that AI can reduce the emissions of buildings.

\begin{enumerate}
    \item By helping scale up the technologies and speed adoption by reducing the construction and labor costs;
    \item By helping reduce emissions in ongoing maintenance and any new construction over the entire building's lifetime.
\end{enumerate}

\subsection*{Scenarios simulated}
The paper uses the results gained from the previous section to \textbf{simulate six scenarios}. The data is used to estimate parameters for use with complex simulation software to attempt to model the potential lifetime impact on emissions.
\begin{enumerate}
    \item Frozen with current building efficiency;
    \item BAU without AI;
    \item BAU with AI;
    \item Three policy-driven scenarios promoting high-efficiency energy buildings and net-zero energy buildings, and other policy implementation to achieve zero emissions by 2050.
\end{enumerate}


\subsection*{Simulation results}
The results of the simulation are shown below.

\begin{table}
\centering
\begin{tabular}{|c|c|c|}
\hline
\textbf{Scenario} & \textbf{Energy Use (kWh/m$^2$)} & \textbf{CO$_2$ Emissions (kg/m$^2$)} \\
\hline 
Baseline & 200 & 50 \\
\hline
AI Optimized & 150 & 30 \\
\hline
\end{tabular}
\caption{Energy use and CO$_2$ emissions for different scenarios.}
\label{tab:energy-emissions}
\end{table}

Table \ref{tab:energy-emissions} shows the energy use and CO$_2$ emissions for different scenarios.

\cfigure{cool-figure}{Different energy use scenarios.}

\subsection*{Key insight}

\textit{“The scenario with AI leads to a higher market share of HEEBs and NZEBs over time compared with the scenario without AI. This trend continues until the market share of net NZEBs reaches its maximum share.”}

The paper asserts that using AI in the ways that they propose always leads to a higher market share of efficient buildings. 


\subsection*{Thoughts on result}
The paper examines various scenarios with some amount of estimation for unknowns, so each individual simulation is unlikely to be exactly right. However, the fact that all scenarios trend in a downward direction for energy use and $CO_2$ emissions suggest it is highly likely that AI would have some impact on building emissions, but the current lack of data leading to necessary estimation means that the current degree of this impact is still unclear.


\section*{Discussion}
\textit{Summary by Mohamed Amine Benaziza}

\subsection*{Method and Scope}
The study uses a combination of engineering and energy-simulation approaches, rather than focusing on a single AI technology, to estimate how AI can boost building energy efficiency and reduce carbon emissions. While the analysis centers on a medium-sized office building, the methodology is adaptable to other commercial building types with appropriate adjustments.

\subsection*{Why AI Helps}
AI enables data-driven modeling, which can tailor solutions to specific buildings and lower costs. This accelerates the adoption of high-efficiency and net-zero-energy buildings (HEEBs and NZEBs). The paper also notes that advanced control models, such as deep learning and reinforcement learning, could further improve accuracy in future work.

\subsection*{Key Analytical Structure}
The paper evaluates the theoretical maximum savings achievable over a building’s lifetime. It highlights that different climate zones offer varying saving potentials, but in all cases, AI can help buildings achieve these potentials at lower costs.

\subsection*{Quantified Impacts (US Medium Offices)}
\begin{itemize}
    \item By 2050, AI adoption alone is projected to reduce energy use and CO$_2$ emissions by approximately 8\% compared to the “business as usual” (BAU) scenario.
    \item When compared to a policy-only scenario, AI provides an additional 19\% in savings.
    \item The combination of AI, strong efficiency policies, and low-emission power generation (LEPG) could result in up to 40\% less energy use and 90\% less CO$_2$ emissions than BAU.
\end{itemize}

\subsection*{Limitations and Future Work}
The results depend on assumptions about cost declines and adoption rates. Additionally, the methodology has not yet been tested on other building types, which may limit the generalizability of the findings.

\subsection*{Take-aways and Thoughts}
\begin{enumerate}
    \item \textbf{Paper’s Goal:} The main contribution is not a new algorithm, but rather the demonstration that AI can make existing high-efficiency designs more affordable, thereby unlocking a larger market share for these technologies.
    \item \textbf{Role of Policy:} On its own, AI delivers modest (single-digit) gains. The substantial reductions of 40\% in energy use and 90\% in CO$_2$ emissions are only achieved when AI is combined with LEPG and clear regulatory policies.
    \item \textbf{Scalability:} Expanding these results to other sectors will require further investigation and the development of skilled facility management teams.
\end{enumerate}

\section*{Methods}
\textit{Summary by James Thompson}


\bibliographystyle{IEEEtran}
\bibliography{references}

\appendix



\end{document}
